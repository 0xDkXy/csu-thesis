%!TEX root = ../../csuthesis_main.tex
\section{总结与展望}

XX的XX都存在XX,所以我们XX,本章总结XX。

\subsection{本文工作总结}
在总结和分析已有XX的理论基础上,本文对XX进行了XX,主要工作如下:

%手工编号
(1)图片插入布局,如第\ref{sec.figure}章所示。

(2)XXXXXXXXXX

(3)XXXXXXXXXX

(4)XXXXXXXXXX

罗马编号
\begin{enumerate}[label=(\roman*)]
 \item XXXXXXXXXX
 \item XXXXXXXXXX
 \item XXXXXXXXXX
\end{enumerate}
括号编号
\begin{enumerate}[label=(\arabic*)]
 \item XXXXXXXXXX
 \item XXXXXXXXXX
 \item XXXXXXXXXX
\end{enumerate}

半括号编号
\begin{enumerate}[label=\arabic*)]
 \item XXXXXXXXXX
 \item XXXXXXXXXX
 \item XXXXXXXXXX
\end{enumerate}

小字母编号
\begin{enumerate}[label=\alph*)]
 \item XXXXXXXXXX
 \item XXXXXXXXXX
 \item XXXXXXXXXX
\end{enumerate}
\subsection{工作展望}
本课题针对XX,鉴于XXX,对XX进行了提高,但是XXX,所以有如下XX:

(1)目前XX虽然XX,但是XX仍然XX,所以XX仍然是一个值得XX的问题。

(2)随着XX,XX具有XX的问题,仍值得进一步XX。

(3)本课题在XX有了XX,但是XX的XX还存在XX,所以XX。

