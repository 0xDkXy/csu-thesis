%!TEX root = ../../csuthesis_main.tex
\chapter{一级标题}

这是中南大学学位论文\LaTeX{}模板,下面的文字主要作用为对重构后的模板样式设置进行测试。
测试样例基本覆盖模板设定,包括多级标题的基本样式,段落与缩进距离。

\section{二级标题}

\subsection{三级标题}

\subsubsection{四级标题}

一级标题根据学校提供的Word模板要求,三号黑体居中,上下各空一行,章节号空一个汉字,
并且每一章节单独起一页,章节号格式应使用阿拉伯数字而非中文汉字。

二级标题为小四号黑体,缩进两个汉字。章节号后空一个汉字。

三级标题小四号楷体GB2312,字体包含在项目中,同样缩进两个汉字,章节号后空一个汉字。

四级标题参照本科学术论文设计样式,分项采取(1)、(2)、(3)的序号。

所有标题样式由\cls{undergraduate.cls}模板文件 \cs{ctexset} 进行设置。

\section{字体}

正文字体默认使用小四号宋体,英文为小四号 Times New Romen,各段行首缩进两个汉字

中南大学坐落在中国历史文化名城──湖南省长沙市,占地面积317万平方米,建筑面积217万平方米,跨湘江两岸,依巍巍岳麓,临滔滔湘水,环境幽雅,景色宜人,是求知治学的理想园地。

中南大学由原湖南医科大学、长沙铁道学院与中南工业大学于2000年4月合并组建而成。原中南工业大学的前身为创建于1952年的中南矿冶学院,原长沙铁道学院的前身为创建于1953年的中南土木建筑学院,两校的主体学科最早溯源于1903年创办的湖南高等实业学堂的矿科和路科。原湖南医科大学的前身为1914年创建的湘雅医学专门学校,是我国创办最早的西医高等学校之一。中南大学秉承百年办学积淀,顺应中国高等教育体制改革大势,弘扬以“知行合一、经世致用”为核心的大学精神,力行“向善、求真、唯美、有容”的校风,坚持自身办学特色,服务国家和社会重大需求,团结奋进,改革创新,追求卓越,综合实力和整体水平大幅提升。

英文字体展示如下:

TeX (/tɛx, tɛk/, see below), stylized within the system as TEX, is a typesetting system (or a "formatting system") which was designed and mostly written by Donald Knuth[1] and released in 1978. TeX is a popular means of typesetting complex mathematical formulae; it has been noted as one of the most sophisticated digital typographical systems.


\section{模板主要结构}

本项目模板的主要结构, 如下表所示:
% TODO

\begin{table}[ht]
  \centering
  \begin{tabular}{r|r|l}
    \hline\hline
    \multicolumn{2}{l|}{Bachelor-template.tex } & 主文档. 在其中填写正文.                                      \\ \hline
                                                & frontmatter.tex                      & 郑重声明、中英文摘要. \\ \cline{2-3}
    \raisebox{1em}{content 目录 }          & backmatter.tex                       & 致谢.                 \\ \hline
    \multicolumn{2}{l|}{figures 目录}         & 存放图片文件.                                                \\ \hline
    \multicolumn{2}{l|}{csuthesis.cls }       & 定义文档格式的 class file. 不可删除.                         \\ \hline\hline
  \end{tabular}
\end{table}

\section{论文组织结构}

全文内容共六章,具体内容组织如下:

第一章为绪论。

第二章为图片插入示例。

第三章为表格插入示例。

第四章为公式插入示例。

第五章为参考文献插入示例。

第六章总结与展望,总结了本文的主要工作,展望了下一阶段的研究方向。
